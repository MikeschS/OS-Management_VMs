\documentclass[10pt,a4paper]{article}
\usepackage[utf8]{inputenc}
\usepackage{amsmath}
\usepackage{amsfonts}
\usepackage{amssymb}
\usepackage{graphicx}
\begin{document}

\huge
\begin{center}
{\bf OS-Management}\par
{\bf Networked Project}\par
\vspace{5cm}
\end{center}

\large
Author: Severin Mikesch\par
\vspace{0.8cm}
Datum: 10.06.2020\par
\vspace{0.8cm}

\normalsize


\newpage
\section{Introduction}
This is a simple guide to show the steps needed to set up virtual machines in Proxmox and install an SSH daemon and NGINX. Furthermore a VPN connection between two VMs will be established and a firewall set up.
\\
I will be using following tools and environments:
\begin{itemize}
\item Proxmox
\item OpenSSH
\item Ubuntu server 20.04 LTS
\item tinc
\item nginx
\item iptables
\end{itemize}

\newpage
\section{Setup of VMs in Proxmox}
Proxmox VE (Virtual Environment) is an open-source solution for enterprise virtualization utilizing the KVM hypervisor running on Debian. I will skip the setup of the Proxmox server as it is a rather long an tedious process and I already have a Proxmox server running at home.
\\
At first we will upload the ISO of Ubuntu server to the proxmox server in order to start our VMs from it.

\subsection{VM Configuration}
Before setting up the VMs we have to make sure that there is a network bridge configured for the LAN in which the client will be placed.
\\
We can leave the defaults of Proxmox when creating the VM. We just have to define following things:
\begin{itemize}
\item ISO Image: ubuntu server
\item Disk Size: 10GB
\item CPU Cores: 2
\item RAM: 2048 MB
\item Network Bridge: your LAN Bridge
\end{itemize}
Now the settings can be confirmed and the VM can be started.

\subsection{Ubuntu server installation}

\subsection{Cloning the VM}

\newpage
\section{SSH Daemon Setup auf VM1}
Example text

\newpage
\section{NGINX Setup}
Example text

\newpage
\section{VPN Setup}
Example text

\newpage
\section{Firewall Setup}
Example text

\end{document}