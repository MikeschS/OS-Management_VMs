\documentclass[10pt,a4paper]{article}
\usepackage[utf8]{inputenc}
\usepackage{amsmath}
\usepackage{amsfonts}
\usepackage{amssymb}
\usepackage{graphicx}
\begin{document}

\huge
\begin{center}
{\bf OS-Management}\par
{\bf Networked Project}\par
\vspace{5cm}
\end{center}

\large
Author: Severin Mikesch\par
\vspace{0.8cm}
Datum: 10.06.2020\par
\vspace{0.8cm}

\normalsize


\newpage
\section{Introduction}
This is a simple guide to show the steps needed to set up virtual machines in Proxmox and install an SSH daemon and NGINX. Furthermore a VPN connection between two VMs will be established and a firewall set up.
\\
I will be using following tools and environments:
\begin{itemize}
\item Proxmox
\item OpenSSH
\item Ubuntu server 20.04 LTS
\item tinc
\item nginx
\item iptables
\end{itemize}

\newpage
\section{Setup of VMs in Proxmox}
Proxmox VE (Virtual Environment) is an open-source solution for enterprise virtualization utilizing the KVM hypervisor running on Debian. I will skip the setup of the Proxmox server as it is a rather long an tedious process and I already have a Proxmox server running at home.
\\
At first we will upload the ISO of Ubuntu server to the proxmox server in order to start our VMs from it.

\subsection{VM Configuration}
Before setting up the VMs we have to make sure that there is a network bridge configured for the LAN in which the client will be placed.
\\
We can leave the defaults of Proxmox when creating the VM. We just have to define following things:
\begin{itemize}
\item ISO Image: ubuntu server
\item Disk Size: 10GB
\item CPU Cores: 2
\item RAM: 2048 MB
\item Network Bridge: your LAN Bridge
\end{itemize}
Now the settings can be confirmed and the VM be started.

\subsection{Ubuntu server installation}
After the VM was started a installer will guide you through the process.

\subsubsection{Language selection}
Select an apropriate language. We will select english.

\subsubsection{optional installer upgrade}
When prompted with an optional installer upgrade this should be accepted. The installer will continue here with the newest version.

\subsubsection{Keyboard layout}
Select the appropriate keyboard layout. We will use the German (Austria) layout.

\subsubsection{Network connections}
Here the previously selected LAN Bridge should be shown. Configure a static IPv4 address. In this example the address 192.168.1.91/24 will be used. 

\subsubsection{Proxy configuration}
If needed a proxy can be defined here. In this example this field will be left blank.

\subsubsection{Archive mirror configuration}
If needed a different archive mirror can be selected. This example will use the default one.

\subsubsection{Storage configuration}
Here the previously defined storage should be shown. In this example the entire disk will be used for the VM. Confirm the selection two times to start writing to the disk.

\subsubsection{Profile setup}
Enter your preferred login credentials and give your vm a name. In this example it will be named virtualmachine1.

\subsubsection{SSH setup}
As we need a ssh daemon on this vm we will install the openssh server here, although we will not import any ssh identities.

\subsubsection{Server snaps}
If needed special services can simply be installed from here. This example will not need any.

\subsubsection{Installation}
Wait for the installation to finish and then reboot.

\subsection{Cloning the VM}

\newpage
\section{SSH Daemon configuration auf VM1}
Example text

\newpage
\section{NGINX Setup}
Example text

\newpage
\section{VPN Setup}
Example text

\newpage
\section{Firewall Setup}
Example text

\end{document}